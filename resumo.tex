Apesar do crescente avan�o no desenvolvimento de algoritmos capazes de induzir modelos preditivos
numa grande diversidade de dom�nios, a participa��o humana ainda � necess�ria em
muitas aplica��es.
A atividade de supervis�o � essencial para a constru��o do conjunto de exemplos
de treinamento que alimenta um algoritmo de aprendizado.
Trata-se de uma atividade custosa, que requer tempo e conhecimento do supervisor.
Consequentemente, � desej�vel que apenas os exemplos mais importantes sejam
considerados.

Existem diferentes formas de se amostrar exemplos relevantes.
Elas pertencem � �rea de aprendizado ativo.
Dada a diversidade existente de abordagens,
� preciso recorrer a algum crit�rio de escolha da melhor estrat�gia,
caso se deseje a melhor efici�ncia poss�vel.
Entretanto, diferentemente da tradicional escolha de algoritmos de aprendizado, que
disp�e dos conjuntos de treinamento para valida��o,
a escolha de estrat�gias de aprendizado ativo � anterior a essa possibilidade.
Assim que uma alternativa � experimentada, o or�amento � gasto,
impedindo o teste de outras alternativas sem incorrer em custos
adicionais.

Este projeto prop�em a investiga��o aprofundada da possibilidade de recomenda��o
autom�tica de estrat�gias por meio de meta-aprendizado,
dado que a �nica tentativa, de conhecimento do candidato, s�o seus
pr�prios experimentos preliminares de conclus�o do doutorado.
Nesses experimentos, h� evid�ncias da viabilidade dessa abordagem, por�m
com implica��es ainda desconhecidas para �rea de aprendizado ativo.
