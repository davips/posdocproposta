Apesar do crescente avanço no desenvolvimento de algoritmos capazes de induzir modelos preditivos
numa grande diversidade de domínios, a participação humana ainda é necessária em
muitas aplicações.
A atividade de supervisão é essencial para a construção do conjunto de exemplos
de treinamento que alimenta um algoritmo de aprendizado.
Trata-se de uma atividade custosa, que requer tempo e conhecimento do supervisor,
frequentemente na atividade de rotulação dos dados.
Consequentemente, é desejável que apenas os exemplos mais importantes sejam
considerados.

A amostragem de exemplos relevantes para rotulação é chamada
de aprendizado ativo e pode ser empreendida segundo diferentes estratégias.
Caso se deseje a melhor eficiência possível,
é preciso recorrer a algum critério de escolha da melhor estratégia.
Entretanto, diferentemente da tradicional escolha de algoritmos de aprendizado, que dispõe do conjunto de treinamento para compará-los por meio de validação,
a escolha de estratégias de aprendizado ativo é anterior a essa possibilidade.
Assim que uma alternativa é experimentada, o orçamento é gasto,
impedindo que outras alternativas sejam testadas sem incorrer em custos
adicionais.

Este projeto propõem a investigação aprofundada da possibilidade de recomendação
automática de estratégias por meio de meta-aprendizado,
dado que a única tentativa, de conhecimento do candidato, são seus
próprios experimentos preliminares de conclusão do doutorado.
Nesses experimentos, em fase de publicação,
há evidências da viabilidade dessa abordagem, porém
com implicações ainda desconhecidas para área de aprendizado ativo.