\section{Proposta}\label{pr}
A literatura atual de aprendizado ativo segue pelo menos duas tendências de criação:
novos cenários, como a rotulação massiva por meio de supervisores de baixo custo \citep{journals/ijcv/VijayanarasimhanG14};
e, novas estratégias como a síntese de exemplos de fronteira \citep{journals/ijon/WangHYL15}
dentre outras.
Apesar das novas propostas de estratégias contribuírem para o avanço da área,
elas tornam a tarefa de escolha de estratégia ainda mais complexa,
pois aumenta a quantidade de opções, sem necessariamente aumentar a eficiência
da amostragem.
Adicionalmente, cada estratégia requer uma implementação compatível com o sistema do usuário,
inviabilizando, assim, a experimentação de grande parte das abordagens existentes.

O candidato do presente projeto comparou, em seu doutorado, estratégias já
consolidadas na literatura e seus nichos de aplicação conforme publicado recentemente
em \cite{santos2014viabilidade} e \cite{conf/hais/SantosC14}.
As possibilidades iniciais de recomendação automática são apresentadas na tese,
ainda em conclusão, antes do fechamento do presente texto.
A proposta deste projeto é manter esse conjunto de estratégias já implementadas
\citep{doi/al}, que é representativo da diversidade de paradigmas existentes, 
como as alternativas disponíveis ao usuário.
Mais precisamente, será considerado cada par estratégia-algoritmo como
uma alternativa.

\textit{O objetivo primário é explorar as possibilidades de recomendação automática
desses pares por meio da aplicação de meta-aprendizado.}

Especificamente, espera-se investigar as seguintes questões:
\begin{compactenum}
\item{Quais são as implicações do uso de meta-aprendizado?}
\item{Ele pode representar uma \textit{quebra de paradigma} na área \citep{kuhn2012structure}?}
\item{Existe um subconjunto ideal de estratégias a se considerar?}
\item{Qual o efeito no meta-aprendizado apenas para agnósticas (ou gnósticas)?}
\item{Qual tipo de recomendação é mais adequado: 
predição de ranqueamento, predição da vencedora, predição de desempenho ou outro?}
\end{compactenum}

\subsection{Infraestrutura}
O trabalho será desenvolvido no Laboratório de Computação Bioinspirada (BIOCOM),
do ICMC/USP.
Ele dispõe de computadores pessoais e servidores para a implementação
e execução de experimentos.
Conta-se ainda com as bibliotecas
das unidades de Computação, Física e Engenharia do Campus de São Carlos.

\subsection{Metodologia}
As técnicas desenvolvidas no projeto serão investigadas
experimentalmente e avaliadas principalmente quanto à viabilidade na recomendação
automática de estratégias.
Serão empregados métodos quantitativos amplamente aceitos na literatura de aprendizado de máquina \citep{Bishop2006} e testes estatísticos
como o teste de Friedman/Nemenyi \citep{journals/jmlr/Demsar06}.

% \subsection{Orçamento}
% Recursos financeiros são necessários para a aquisição de material permanente,
% passagens, diárias e material bibliográfico.
% O material permanente se refere aos equipamentos necessários para incrementar a
% base de computadores instalados no laboratório do grupo de pesquisa.
% As passagens e diárias viabilizarão a participação em conferências e congressos relevantes à linha de pesquisa deste projeto.
% O material bibliográfico contribuirá para a atualização do acervo bibliográfico do
% instituto.

% \begin{table}[!htp]
% \centering
% \begin{tabular}{|lr|}
% \hline
% \textbf{Tipo do recurso} & \textbf{Custo total (R\$)} \\
% \hline
% \multicolumn{2}{|c|}{Ano 1} \\
% \hline
% Material permanente & 3.500,00 \\
% Passagens           & 6.000,00 \\
% Diárias             & 1.500,00 \\
% Material bibliográfico & 1.000,00 \\
% \textbf{Total} & 12.000,00 \\
% \hline
% \multicolumn{2}{|c|}{Ano 2} \\
% \hline
% Material permanente & 3.500,00 \\
% Passagens           & 6.000,00 \\
% Diárias             & 1.500,00 \\
% Material bibliográfico & 1.000,00 \\
% \textbf{Total} & 12.000,00 \\
% \hline
% \end{tabular}
% \caption{Custos do projeto.}
% \end{table}

\subsection{Plano de atividades}
As atividades previstas para a execução deste projeto de pós-doutorado são as seguintes:

\begin{compactenum}

\item{\textbf{Pesquisa bibliográfica.} Pesquisa bibliográfica com o objetivo de
monitorar o surgimento de eventuais publicações que conciliem aprendizado ativo e meta-aprendizado. Atualização de conhecimentos e estudo de alternativas para cumprir os objetivos apontados.}

\item{\textbf{Implementação de técnicas.} A partir da pesquisa bibliográfica, é possível que novas estratégias se mostrem interessantes do ponto de 
vista de sua representatividade e contribuição para a diversidade de opções
do sistema de recomendação. Elas podem ser implementadas para integrar o sistema de meta-aprendizado.}

\item{\textbf{Experimentação e validação.} Por meio da metodologia comentada anteriormente serão conduzidos experimentos para avaliação das propostas.}

\item {\textbf{Redação do relatório final.} Redação do relatório final expondo os resultados obtidos no decorrer deste projeto.}

\item{\textbf{Elaboração de artigos científicos.} Serão redigidos artigos que reflitam o grau de desenvolvimento do projeto e as contribuições realizadas.}
\end{compactenum}

\subsection{Cronograma}\label{ssec:Cronograma}
O cronograma de desenvolvimento das atividades compreende dois anos.
Ele é apresentado na Tabela \ref{cro}.

\newcommand{\y}{\rule{13pt}{5pt}}
\newcommand{\x}{\hspace*{4pt}}
\setlength{\tabcolsep}{0pt}

\begin{table}[h] %\footnotesize
\centering
\begin{tabular}{|l|c|c|c|c|c|c|c|c|c|c|c|c|}

  \cline{2-13}
  \multicolumn{1}{l|}{} &
  \multicolumn{6}{c|}{\rotatebox{90}{Ano 1\hspace{3pt}}} &
  \multicolumn{6}{c|}{\rotatebox{90}{Ano 2\hspace{3pt}}} \\

  \cline{2-13}
%    \multicolumn{1}{c|}{\textbf{Meses}} \\

  \multicolumn{1}{c|}{\textbf{Atividades}} &
                        \rotatebox{90}{01 - 02\hspace{6pt}} &
                        \rotatebox{90}{03 - 04\hspace{6pt}} &
                        \rotatebox{90}{05 - 06\hspace{6pt}} &
                        \rotatebox{90}{07 - 08\hspace{6pt}} &
                        \rotatebox{90}{09 - 10\hspace{6pt}} &
                        \rotatebox{90}{11 - 12\hspace{6pt}} &
                        \rotatebox{90}{01 - 02\hspace{6pt}} &
                        \rotatebox{90}{03 - 04\hspace{6pt}} &
                        \rotatebox{90}{05 - 06\hspace{6pt}} &
                        \rotatebox{90}{07 - 08\hspace{6pt}} &
                        \rotatebox{90}{09 - 10\hspace{6pt}} &
                        \rotatebox{90}{11 - 12\hspace{6pt}} \\
  \hline
  1)Pesquisa bibliográfica & \y & \y & \y & \y & \y & \y & \y & \y & \y & \y & \y & \y  \\
  \hline
  2)Implementação & \x & \x & \x & \x & \y & \y & \y & \y & \y & \x & \x & \x  \\
  \hline
  3)Experimentação e validação & \x & \x & \x & \x & \x & \y & \y & \y & \y & \y & \x & \x  \\
  \hline
  4)Redação do relatório & \x & \x & \x & \x & \x & \x & \x & \x & \x & \x & \x & \y  \\
  \hline
  5)Elaboração de artigos & \x & \x & \x & \x & \x & \x & \y & \y & \y & \y & \y & \y  \\
  \hline
\end{tabular}
 \caption{Cronograma de atividades}
 \label{cro}
 \normalsize
\end{table}